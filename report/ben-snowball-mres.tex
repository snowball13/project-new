\documentclass[11pt, oneside]{article}   	% use "amsart" instead of "article" for AMSLaTeX format
\usepackage{geometry}                		% See geometry.pdf to learn the layout options. There are lots.
\geometry{letterpaper}                   		% ... or a4paper or a5paper or ... 
%\geometry{landscape}                		% Activate for rotated page geometry
\usepackage[parfill]{parskip}    		% Activate to begin paragraphs with an empty line rather than an indent
\usepackage{graphicx}				% Use pdf, png, jpg, or eps§ with pdflatex; use eps in DVI mode
								% TeX will automatically convert eps --> pdf in pdflatex		
\usepackage{amssymb}
\usepackage{amsmath}
\usepackage{graphicx}
\usepackage{bbm}

%SetFonts

%SetFonts

\newcommand{\R}{\mathbb{R}}
\newcommand{\MN}{\mathbb{M}_N}
\newcommand{\LN}{\mathbb{L}_N}
\newcommand{\dsm}{d_\mathbb{S}(m)}
\newcommand{\dsmsq}{d^{2}_{\mathbb{S}}(m)}
\newcommand{\graddsmsq}{\nabla{d^{2}_{\mathbb{S}}(m)}}
\newcommand{\eone}{\hat{e}_1}
\newcommand{\etwo}{\hat{e}_2}
\newcommand{\bt}{\tilde{b}}
\newcommand{\pt}{\tilde{p}}
\newcommand{\dt}{\Delta t}
\newcommand{\M}{\mathbb{M}}
\newcommand{\N}{\mathbb{N}}
\newcommand{\Ps}{\mathbb{P}_{\mathbb{S}}}
\newcommand{\Pm}{\mathbb{P}_{\mathbb{M}_N}}
\newcommand{\Leb}{\mathrm{Leb}}
\newcommand{\DmDt}{\frac{\mathrm{D}m}{\mathrm{D}t}}
\newcommand{\DuDt}{\frac{\mathrm{D}u}{\mathrm{D}t}}


\title{MRes Report}
\author{Ben Snowball}
%\date{}							% Activate to display a given date or no date


\begin{document}



\maketitle
\section{Introduction}

The main goal of this paper is to investigate whether using an optimal transport method can be used to simulate our Eady model. We extend the method, introducing notation and our set of differential equations for our vertical slice Eady model, and detail how our program works. We finally present some results and conclusions.


\section{The Eady Model}

Here, we describe the process of obtaining our vertical slice Eady model. Let \(m\) be the 2-dimensional position vector in the \(x,z\text{-plane}\), and let \(u\) be the 2-dimensional velocity vector. The simple model describing the incompressible flow is:
\begin{align}
\DmDt &= u, \\
\DuDt &= -\nabla p, 
\end{align}
where \(\frac{\mathrm{D}}{\mathrm{D}t}\) is the material derivative, and \(p\) is some pressure term. We add the constraint that we wish the distance from our point in space mapping \(m\) to the set of measure preserving maps \(\mathbb{S}\) to be zero. This means that in order to have a conserved Hamiltonian we need to incorporate this into it....



\section{Extension of the Lagrangian Euler Scheme to the Eady Model}

Our domain for our problem is \(\Omega = [-L, L] \times [-H, H] \subset \R^2\) with periodic boundary conditions in the \(x\)-direction. Let \(\M := L^2(\Omega, \R^2)\). We define \(\mathbb{S}\) as the set of volume preserving maps from \(\Omega \to \R^2\), that is:

\begin{align}
\mathbb{S} &= \{ s \in \M \quad | \quad s_{\#}\Leb(A) := \Leb(s^{-1}(A)) = \Leb(A) \quad \forall A \subset \Omega \},
\end{align}

so \(\mathbb{S} \subset \M\). Now, for \(N \in \N\), we define a tessellation partition \(P_N\) of \(\Omega\) into \(N\) subsets \(\omega_i, i=1,\dots,N\) such that:

\begin{align}
1)& \quad \Leb(\omega_i) = \frac{1}{N}\Leb(\Omega) \quad \forall i = 1,\dots,N \\
2)& \quad \max_{i = 1,\dots,N} \mathrm{diam}(\omega_i) \le CN^{-\frac{1}{2}}, \quad C \text{ independent of } N.
\end{align}

Then we define \(\MN\) to be the space of functions from \(\Omega\) to \(\R^2\) which are constant on each of the subdomains \(\omega_i\) of \(P_N\), and \(\LN\) to be the space of functions from  \(\Omega\) to \(\R\) which are constant on each of the subdomains of \(P_N\) also. i.e.

\begin{align}
\MN &= \{ \phi \: : \: \Omega \to \R^2 \quad | \quad \phi | _{\omega_i} \text{ is constant, } i = 1,\dots,N\}, \\
\LN &= \{ a \: : \: \Omega \to \R \quad | \quad a | _{\omega_i} \text{ is constant, } i = 1,\dots,N\}.
\end{align}

We note the meaning of the (squared) distance from a map in \(\MN\) to the set \(\mathbb{S}\) is given by \(\dsmsq \: : \: \MN \to \R\) where
\begin{align}
\dsmsq &:= \min_{s \in \mathbb{S}} || m - s || ^2_{L^2}, \quad m \in \MN.
\end{align}

Since \(\mathbb{S}\) is closed but not convex, the orthogonal projection of a map in \(\mathbb{S}\) exists, but is not uniquely defined, and so is simply given as any map that is a projection. In other words, a projection of \(m \in \mathbb{M}\), \(P_\mathbb{S}(m)\), is any point satisfying
\begin{align}
|| P_\mathbb{S}(m) - m || &= \dsmsq,
\end{align}

and denote the orthogonal projection mapping on \(\MN\) by \(P_{\MN}\).

We choose a set of variables \(z = (m, u, v, b) \in \MN \times \MN \times \LN \times \LN\), that satisfy the system of equations:
\begin{align} 
\dot{m} &= u, \\
\dot{u} &= f v \eone - \nabla p + b \etwo, \\
\dot{v} &= - f u_1 - s(m_2 - H/2), \\
\dot{b} &=  -vs,
\end{align}

where \(\nabla p = \frac{\graddsmsq}{\epsilon^2}\),  \(f\) is the coriolis force constant, \(s = \frac{\partial{b}}{\partial{y}}\) is the (constant) gradient of the bouyancy in the \(y\)-direction, and \(\epsilon\) is the spring parameter. Here, \(\eone\) and \(\etwo\) are the unit vectors in the \(x\) and \(z\) directions respectively.

From (reference for Quentin 1), a calculatable formula for \(\graddsmsq\) exists. First, we introduce the definition of a Laguerre diagram. Let \(M = (M_1,\dots,M_N), \text{ where } M_i \in \R^2, \psi = (\psi_1,\dots,\psi_N), \text{ where } \psi_i \in \R\). The Laguerre diagram is a decomposition of \(\R^2\) into \(N\) convex polyhedra (Laguerre cells) defined by:
\begin{align}
\mathrm{Lag}_i(M, \psi) &= \{x \in \R^2 \quad | \quad || x - M_i ||^2 + \psi_i \le || x - M_j ||^2 + \psi_j, \quad \forall j = 1,\dots,N\}
\end{align}

Proposition 1: Let \(m \in \MN\) and define \(M_i := m(\omega_i) \in \R^2 \text{ for } i = 1,\dots,N\) with \(M = (M_1,\dots,M_N)\). Then there exists \(\psi = (\psi_1,\dots,\psi_N), \psi \in \R\), unique up to additive constant such that, with \(L_i := \mathrm{Lag}_i(M, \psi)\), 
\begin{align}
\Leb(L_i) &= \frac{1}{N}\Leb(\Omega),
\end{align}
and
\begin{align}
\graddsmsq &= 2(m - P_{\MN} \circ P_{\mathbb{S}}(m)),
\end{align}
where
\begin{align}
P_{\MN} \circ P_{\mathbb{S}}(m) &= \sum^N_{i=1} B_i \mathbbm{1}_{L_i}(m), \\
B_i &:= \frac{1}{\Leb(L_i)} \int_{L_i} x \: \mathrm{d}x, \quad i = 1,\dots,N. 
\end{align}



\section{Nondimensionalistaion}

We nondimensionalise the equations to help with the computational implementation. We first set:
\begin{align} 
m_1 &= X m_1', \\
m_2 &= Z m_2', \\
u &= u_0 u', \\
v &=  u_0 v', \\
b &= b_0 b', \\
s &= \frac{b_0}{Y} s', \\
f &= f_0 f' \\
t &= \frac{L}{u_0} t',
\end{align}

where the primes denote nondimensional parameters. Note that we can simply set \(f' = 1\). We then have, 

\begin{equation}
\frac{\mathrm d}{\mathrm d t} = \frac{\mathrm d t'}{\mathrm d t} \frac{\mathrm d}{\mathrm d t'} = \frac{u_0}{L} \frac{\mathrm d}{\mathrm d t'}.
\end{equation}

Using these, we can then nondimensionalise our original equations (note that in the following, \(\dot{a} = \frac{\mathrm d a}{\mathrm d t'}\) where \(a\) is any of \(m', u', v', b'\)):
\begin{align} 
\dot{m'} &= \frac{L}{X} u_1' \eone +  \frac{L}{Z} u_2' \etwo, \\
\dot{u'} &= \frac{L f}{u_0} v' \eone - \nabla \tilde{p} + \frac{L b_0}{u_0^2} b' \etwo, \\
\dot{v'} &= -\frac{L f}{u_0} u_1' - (\frac{Z}{Y} m_2' - \frac{H}{2L}) \frac{L b_0}{u_0^2} s', \\
\dot{b'} &= -\frac{L}{Y} v' s'.
\end{align}

We can then choose appropriate values for the dimensional constants as follows:
\begin{align} 
X &= Y = L, \\
Z &= H, \\
b_0 &= N^2 H, \\
u_0 &= \frac{N H}{10},
\end{align}

where \(N\) is the B-V frequency. We can then define the Rossby and Froude numbers as
\begin{align} 
Ro &= \frac{u_0}{L f}, \\
Fr &= \frac{u_0}{N H}.
\end{align}

This gives the nondimensional equations (note for convenience we now drop the primes and the tilde):
\begin{align} 
\dot{m} &= u_1 \eone +  \frac{L}{H} u_2 \etwo, \\
\dot{u} &= \frac{1}{Ro} v \eone - \nabla p + \alpha b \etwo, \\
\dot{v} &= -\frac{1}{Ro} u_1 - (m_2 - \frac{1}{2}) \frac{1}{Fr^2} s, \\
\dot{b} &= -v s,
\end{align}
where we define
\begin{align}
\alpha := \frac{L}{H Fr^2}.
\end{align}

To avoid the process of ensuring hydrostatic balance on initilisation, we actually only need to look at the perturbation of the buoyancy \(b\), which we denote \(\bt\), and similarly the perturbation of the pressure term \(p\), which we denote \(\pt\), i.e.
\begin{align} 
b(x, t) &= b(x, 0) + \bt(x, t),
\end{align}
where
\begin{align} 
b(x, 0) &=: b^{(0)} = m_2 - \frac{1}{2}.
\end{align}

Thus,
\begin{align} 
\dot{b} &= u.\nabla b^{(0)} + \dot{\bt} \\
\implies \dot{\bt} &= \dot{b} - u.\nabla b^{(0)} \\
\implies \dot{\bt} &= -vs - u_2.
\end{align}

Further, for the pressure term, we set
\begin{align} 
p =  p_h + \pt,
\end{align}
and thus
\begin{align} 
\dot{u} &= \frac{1}{Ro} v \eone - \nabla p_h - \nabla \pt + \alpha b^{(0)} \etwo + \alpha \bt \etwo.
\end{align}

Upon setting
\begin{align} 
p_h = p^{(0)} + \int_{z_0}^z \alpha b^{(0)}\,\mathrm{d}\xi
\end{align}
we have, as \(b^{(0)}\) is independent of \(x\), that \(\nabla p_h . \eone = 0\) and so
\begin{align} 
\nabla p_h = \alpha b^{(0)} \etwo.
\end{align}

Thus, our nondimensional equations become
\begin{align} 
\dot{m} &= u_1 \eone +  \frac{L}{H} u_2 \etwo, \\
\dot{u} &= \frac{1}{Ro} v \eone - \nabla \pt + \alpha \bt \etwo, \\
\dot{v} &= -\frac{1}{Ro} u_1 - (m_2 - \frac{1}{2}) \frac{1}{Fr^2} s, \\
\dot{\bt} &= -v s - u_2,
\end{align}

with the initial conditions
\begin{align} 
\bt |_{t=0} &= 0, v |_{t=0} = 0, u_2 |_{t=0} = 0, \\
u_1 |_{t=0} &= -s (m_2 - \frac{1}{2}) \frac{Ro}{Fr^2} \quad \text{(geostrophic balance)}.
\end{align}



\section{Timestepping}

We use a splitting method to execute a timestepping procedure to solve equations (38-41). The equations corresponding to the kinetic energy are:
\begin{align} 
\dot{m} &= u_1 \eone +  \frac{L}{H} u_2 \etwo, \\
\dot{u} &= \frac{1}{Ro} v \eone, \\
\dot{v} &= -\frac{1}{Ro} u_1, \\
\dot{\bt} &= -v s.
\end{align}

which has the exact solution of:
\begin{align} 
m^{n+1} &= m^n + \dt \frac{L}{H} u_2 \etwo + Ro [\sin(\frac{\dt}{Ro}) u_1^n - (\cos(\frac{\dt}{Ro}) - 1) v^n] \eone, \\
u^{n+1} &= u_2^n \etwo + [\cos(\frac{\dt}{Ro}) u_1^n - \sin(\frac{\dt}{Ro}) v^n] \eone, \\
v^{n+1} &= \sin(\frac{\dt}{Ro}) u_1^n - \cos(\frac{\dt}{Ro}) v^n, \\
\bt^{n+1} &= \bt^n - sRo[\sin(\frac{\dt}{Ro}) v^n - (\cos(\frac{\dt}{Ro}) - 1) u_1^n].
\end{align}

Further, the equations corresponding to the potential energy are:
\begin{align} 
\dot{m} &= 0, \\
\dot{u} &= - \nabla \pt + \alpha \bt \etwo, \\
\dot{v} &= - (m_2 - \frac{1}{2}) \frac{s}{Fr^2}, \\
\dot{\bt} &= -u_2,
\end{align}

which has the exact solution of:
\begin{align} 
m^{n+1} &= m^n, \\
u^{n+1} &=  u_1^n \eone - \dt \nabla \pt(m^n) + [\sqrt{\alpha} \sin(\sqrt{\alpha} \dt)  \bt^n + \cos(\sqrt{\alpha} \dt) u_2^n] \etwo, \\
v^{n+1} &= v^n - \dt (m_2^n - \frac{1}{2})\frac{s}{Fr^2}, \\
\bt^{n+1} &= \cos(\sqrt{\alpha} \dt)  \bt^n + \frac{1}{\sqrt{\alpha}} \sin(\sqrt{\alpha} \dt) u_2^n.
\end{align}

The two exact solutions can be combined to make a Hamiltonian splitting method, of which the lowest order scheme is the non-canonical symplectic Euler method:
\begin{align} 
m^{*} &= m^n + \dt \frac{L}{H} u_2 \etwo + Ro [\sin(\frac{\dt}{Ro}) u_1^n - (\cos(\frac{\dt}{Ro}) - 1) v^n] \eone, \\
u^{*} &= u_2^n \etwo + [\cos(\frac{\dt}{Ro}) u_1^n - \sin(\frac{\dt}{Ro}) v^n] \eone, \\
v^{*} &= \sin(\frac{\dt}{Ro}) u_1^n - \cos(\frac{\dt}{Ro}) v^n, \\
\bt^{*} &= \bt^n - sRo[\sin(\frac{\dt}{Ro}) v^n - (\cos(\frac{\dt}{Ro}) - 1) u_1^n], \\
m^{n+1} &= m^{*}, \\
u^{n+1} &=  u_1^{*} \eone - \dt \nabla \pt(m^{*}) + [\sqrt{\alpha} \sin(\sqrt{\alpha} \dt)  \bt^{*} + \cos(\sqrt{\alpha} \dt) u_2^{*}] \etwo, \\
v^{n+1} &= v^{*} - \dt (m_2^{*} - \frac{1}{2})\frac{s}{Fr^2}, \\
\bt^{n+1} &= \cos(\sqrt{\alpha} \dt)  \bt^{*} + \frac{1}{\sqrt{\alpha}} \sin(\sqrt{\alpha} \dt) u_2^{*},
\end{align}

where \(\alpha = \frac{L}{H Fr^2}\).



\section{Timestepping for Full Scheme}

After more careful thought, we discovered that nondimensionalising the equations meant we lost the measure preserving properties of the projections in the optimal transport method. Thus, we instead return to our original equations, while still looking at the perturbation of the buoyancy \(b\) from some background state.

To avoid the process of ensuring hydrostatic balance on initilisation, we actually only need to look at the perturbation of the buoyancy \(b\), which we denote \(\bt\), and similarly the perturbation of the pressure term \(p\), which we denote \(\pt\), i.e.
\begin{align} 
b(x, t) &= b(x, 0) + \bt(x, t),
\end{align}
where
\begin{align} 
b(x, 0) &=: b^{(0)} = N^2 (m_2 - \frac{H}{2}).
\end{align}

Thus,
\begin{align} 
\dot{b} &= u.\nabla b^{(0)} + \dot{\bt} \\
\implies \dot{\bt} &= \dot{b} - u.\nabla b^{(0)} \\
\implies \dot{\bt} &= -vs - N^2 u_2.
\end{align}

Further, for the pressure term, we set
\begin{align} 
p =  p_h + \pt,
\end{align}
and thus
\begin{align} 
\dot{u} &= f v \eone - \nabla p_h - \nabla \pt + b^{(0)} \etwo + \bt \etwo.
\end{align}

Upon setting
\begin{align} 
p_h = p^{(0)} + \int_{z_0}^z b^{(0)}\,\mathrm{d}\xi
\end{align}
we have, as \(b^{(0)}\) is independent of \(x\), that \(\nabla p_h . \eone = 0\) and so
\begin{align} 
\nabla p_h = b^{(0)} \etwo.
\end{align}

Thus, our full set of differential equations to solve are:

\begin{align} 
\dot{m} &= u, \\
\dot{u} &= f v \eone - \nabla \pt + \bt \etwo, \\
\dot{v} &= -f u_1 - (m_2 - \frac{H}{2}) s, \\
\dot{\bt} &= -v s - N^2 u_2.
\end{align}

To solve, we use a Runge-Kutta method of order 4 (RK4), implemented as follows. We wish to solve:

\begin{align} 
\dot{x} &= h(x)
\end{align}

where we set
\begin{align} 
x &= (m_1, m_2, u_1, u_2, v, \bt), \\
h(x) :&= (u_1, u_2, fv + (\nabla \pt)_1, \bt + (\nabla \pt)_2, -fu_1 -s(m_2 - \frac{H}{2}), -sv -N^2u_2).
\end{align}

Then at each timestep, we calculate:
\begin{align} 
x^{n+1} &= \frac{\dt}{6}(k_1 + 2k_2 + 2k_3 + k_4),
\end{align}

using

\begin{align} 
k_1 &= h(x^n), \\
k_2 &= h(x^n + \frac{\dt}{2}k_1), \\
k_3 &= h(x^n + \frac{\dt}{2}k_2), \\
k_4 &= h(x^n + \dt k_3).
\end{align}

We also provide a "kick" to the perturbation of the buoyancy on initialisation, to induce the instability.


\section{Initialising the Grid}

We begin by taking a set of uniformly spaced points, as an initial setup for our particles. We then perturb these slightly away from the uniformity, and execute a damped Newton algorithm to move our points into an equispaced setting (solve the optimal transport problem????).




\end{document}











  